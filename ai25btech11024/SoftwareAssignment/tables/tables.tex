\let\negmedspace\undefined
\let\negthickspace\undefined
\documentclass[journal]{IEEEtran}
\usepackage[a5paper, margin=10mm, onecolumn]{geometry}
%\usepackage{lmodern} % Ensure lmodern is loaded for pdflatex
\usepackage{tfrupee} % Include tfrupee package

\setlength{\headheight}{1cm} % Set the height of the header box
\setlength{\headsep}{0mm}     % Set the distance between the header box and the top of the text

\usepackage{gvv-book}
\usepackage{gvv}
\usepackage{cite}
\usepackage{amsmath,amssymb,amsfonts,amsthm}
\usepackage{algorithmic}
\usepackage{graphicx}
\usepackage{textcomp}
\usepackage{xcolor}
\usepackage{txfonts}
\usepackage{listings}
\usepackage{enumitem}
\usepackage{mathtools}
\usepackage{gensymb}
\usepackage{comment}
\usepackage[breaklinks=true]{hyperref}
\usepackage{tkz-euclide} 
\usepackage{listings}
% \usepackage{gvv}                               

\def\inputGnumericTable{}                      
\usepackage[latin1]{inputenc}                    
\usepackage{color}                              
\usepackage{array}                             
\usepackage{longtable}                          
\usepackage{calc}                               
\usepackage{multirow}                           
\usepackage{hhline}                            
\usepackage{ifthen}                          
\usepackage{lscape}
\begin{document}

\bibliographystyle{IEEEtran}
\vspace{3cm}

\title{Error Analysis and Tables}
\author{AI25BTECH11024 - Pratyush Panda
}
\maketitle
% \newpage
% \bigskip
{\let\newpage\relax\maketitle}

\renewcommand{\thefigure}{\theenumi}
\renewcommand{\thetable}{\theenumi}
\setlength{\intextsep}{10pt} % Space between text and floats


\numberwithin{equation}{enumi}
\numberwithin{figure}{enumi}
\renewcommand{\thetable}{\theenumi}

\section*{Error Analysis}
The error being calculated are Frobenius Error which is defined as: \\
For the given matrix $\Vec{A}_i$, and the resultant matrix $\Vec{A}_f$ the error can be written as:
\begin{align}
    e = ||\Vec{A}_f - \Vec{A}_i||_F
\end{align}

Error for the globe picture:
\begin{table}[h!]
    \centering
    \begin{tabular}{c|c}
    k Value & Frobenius Error \\
    \hline
    50 & 24.550934 \\
    90 & 16.140119 \\
    150 & 10.328848 \\
    200 & 7.526983 \\
    \end{tabular}
    \caption{img 1}
    \label{img 1}
\end{table}

Error for the gray-scale image:
\begin{table}[h!]
    \centering
    \begin{tabular}{c|c}
    k Value & Frobenius Error \\
    \hline
    50 & 4.666718 \\
    90 & 1.982175 \\
    150 & 1.170241 \\
    200 & 0.789041 \\
    \end{tabular}
    \caption{img 2}
    \label{img 2}
\end{table}

Error for the Einstein image:
\begin{table}[h!]
    \centering
    \begin{tabular}{c|c}
    k Value & Frobenius Error \\
    \hline
    50 & 3.699289 \\
    90 & 1.067723 \\
    150 & 0.039846 \\
    200 & 0.176103 \\
    \end{tabular}
    \caption{img 3}
    \label{img 3}
\end{table}

In general, after high values in k, there was not a lot of changes in the image, but for small values of k, the image quality did become very poor. Frobenius Error is a good way to judge the difference between the image quality. We can see that when we move from k=50 to k=90 the error decreases quite a lot, while when we go from k=150 to k=200, the change in error is not that much. Thus we can say, for large values of k, the image quality does not improve much and converges to the original image.


\end{document}
